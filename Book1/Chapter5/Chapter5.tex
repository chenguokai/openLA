\documentclass{ctexart}
\usepackage{geometry}
\usepackage{amsmath}
\usepackage{amsfonts}

\begin{document}
\section{代数学引论  第一卷}
\section{第五章  复数和多项式}
\subsection{§1  复数域}
\subsubsection{§1.1 辅助结构}
\paragraph{定义}
$\mathbb{C} =  \{ a + b\sqrt{-1} \mid a, b \in \mathbb{R} \}$
在本节中$i$代表$\sqrt{-1}$
\paragraph{定义}
设 $a$, $b \in \mathbb{R}$,$z = a + bi$,$z$的共轭是 $a - bi$,记为$\bar{z}$
\paragraph{引理 1.1}
设 $z \in \mathbb{C}$,则\\$(i) z\bar{z} \in \mathbb{R}\\ (ii) z \ne 0\Leftrightarrow z\bar{z} > 0$
\paragraph{命题 1.1}
\rule[8pt]{0.3cm}{0.05em} : 
$\begin{aligned} 
\mathbb{C} \rightarrow \mathbb{C} \\
z \mapsto \bar{z}
\end{aligned}$
 是同构
 \subsubsection{§1.2 复数的几何解释}
 \paragraph{命题1.2}
 $(i)$设$z$_{1}$ = |$z$$_{1}$|(\cos\theta$_{1}$+$i$\sin\theta$_{1}$),$z$_{2}$ = |$z$$_{2}$|(\cos\theta$_{2}$+$i$\sin\theta$_{2}$)\\
 则 $z$$_{1}$$z$$_{2}$ = |$z$$_{1}$||$z$$_{2}$|(\cos(\theta$_{1}$+\theta$_{2}$)+$i$\sin(\theta$_{1}$+\theta$_{2}$))\\
 $(ii)$设$z = |z|(\cos\theta+i\sin\theta)$,$n \in \mathbb{N}$,则 $z$^n$ = |$z$$^n$|(\cos n\theta+i\sin n\theta)$\\
 $(iii)$设$z$同$(ii)$,$z \ne 0$,则$z$$^{-1}$$ = \frac{1}{|z|}$$(\cos(-\theta)+i\sin(-\theta))$\\
\paragraph{命题1.3}
方程$z$$^{n}$$-1=0$ ($n$ \in \mathbb{Z}$^{+}$) 有$n$个不同的复数解\\ \varepsilon$_{k}$ = \cos\frac{2k\pi}{n} + i\sin\frac{2k\pi}{n}\quad k$ = 0,1,...,$n - 1$
\paragraph{定义}
称上述\varepsilon$_{0}$,\varepsilon$_{1}$,...,\varepsilon$_{n-1}$为$n$次单位根
\paragraph{定义}
设\varepsilon$_{k}$ $\in U$_{n}$,如果$U$$_{n}$ = <\varepsilon$_{k}$>,则称\varepsilon$_{k}$是$n$次本原单位根
\end{document}