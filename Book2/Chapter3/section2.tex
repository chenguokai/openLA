\ifx\total\undefined 
\documentclass{ctexart}
\usepackage{geometry}
\usepackage{amsmath}
\usepackage{amsfonts}
\begin{document}
\fi

\paragraph{定义}
设$\mathcal{A} \in \mathcal{L}(V),\mathcal{A}^{*}$是$\mathcal{A}$的伴随算子,如果$\mathcal{A} \circ \mathcal{A}^{*}$,则称$\mathcal{A}$是正规(normal)算子.设$\mathcal{A} \in M_{n}(\mathbb{R})$,如果$AA^{t} = A^{t}A$,则称$A$是正规矩阵.

\paragraph{注}
由定理2.1和第二章定理2.1可知,$\mathcal{A}$是正规算子当且仅当$\mathcal{A}$在某组单位正交基下的矩阵是正规的.

\paragraph{引理2.1}
设$\mathcal{A} \in \mathbb{R}^{m \times n}$,如果$tr(AA^{t}) = 0$,则$A = O_{m \times n}$

\paragraph{引理2.2}
设$W$是$\mathcal{R}$上$n$维线性空间,$n>0$,$\mathcal{A} \in \mathcal{L}(V)$,则$W$有1维或2维不变子空间.

\paragraph{引理2.3}
设$A \in M_{n}(\mathbb{R})$是正规的,如果
%matrix begin
$$
 A =
 \left[
 \begin{matrix}
  A_{1} & A_{2} \\
   0 & A_{3}
  \end{matrix}
  \right]
$$,其中$A_{1} \in M_{d}(\mathbb{R}),A_{2} \in \mathbb{R}^{d \times n-d},A_{3} \in M_{n-d}(\mathbb{R}),0<d<n.$,则$A_{2} = 0$
%matrix end

\paragraph{引理2.4}
设$\mathcal{A} \in \mathcal{L}(V)$正规,如果$U \subset V$是$\mathcal{A}-$不变子空间,则$U^{\bot}$也是

\paragraph{引理2.5}
设$\mathcal{A} \in \mathcal{L}(V)$正规,则存在$A-$不可分子空间$U_{1},\cdots,U_{l}$使得\\
(i) $V = U_{1} \bigoplus \cdots \bigoplus U_{l}$
(ii) $\forall i,j \in \{1,\cdots\}$

\ifx\total\undefined 
\end{document}
\fi
