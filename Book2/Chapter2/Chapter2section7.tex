\ifx\total\undefined
\documentclass{ctexart}
\usepackage{geometry}
\usepackage{amsmath}
\usepackage{amsfonts}
\begin{document}
\fi

\subsection{§7 复矩阵的Jordan标准型(存在性)}
\paragraph{定理7.1}
设$\mathcal{A} \in M_{n}(\mathbb{C})$,则存在$\lambda_{1},\cdots,\lambda_{l} \in \mathbb{C}$(不必两两不同),$d_{1},\cdots,d_{l} \in \mathbb{Z}^{+}$,使得
%matrix begin
$$
 A\sim_{s}J_{\mathcal{A}}=
 \left[
 \begin{matrix}
  J_{d_{1}}(\lambda_{1}) &   &   & 0 \\
    & \ddots &   &   & \\
     &   & \ddots &   &  \\
  0 &   &    & J_{d_{l}}(\lambda_{l}) 
  \end{matrix}
  \right]_{n\times n}
$$
%matrix end

\paragraph{注}
(1) $\lambda_{1},\cdots,\lambda_{l}$互不相同的元素组成$spce_{\mathbb{C}}(A)$\\
(2) $\mathcal{X}_{A}=(t-\lambda_{1})^{d_{1}}\cdots(t-\lambda_{l})^{d_{l}}$\\
$\mu_{A}=lcm((t-\lambda_{1})^{d_{1}}\cdots(t-\lambda_{l})^{d_{l}})$\\
(3) 如果$d_{1}=\cdots=d_{l}=1$,则
%matrix begin
$$
 J_{\mathcal{A}}=
 \left[
 \begin{matrix}
  \lambda_{1} &   &   & 0 \\
    & \ddots &   &   & \\
     &   & \ddots &   &  \\
  0 &   &    & \lambda_{l}
  \end{matrix}
  \right]_{n\times n}
$$
%matrix end
即$A$可对角化

\paragraph{注}
$d_{1},\cdots,d_{l}$是否唯一?$l$是否唯一?即$J_{A}$是否唯一

\subsection{§8 矩阵的准素有理规范型简介}

\paragraph{定理8.1}
设$A\in M_{n}(F)$,则存在$d_{1},\cdots,d_{l}\in \mathbb{Z}^{+}$,则存在$l_{1},\cdots,l_{s} \in \mathbb{Z}^{+},p_{1},\cdots,p_{s} \in F[t]\backslash F$不可约,使得
%matrix begin
$$
 A \sim
 \left[
 \begin{matrix}
  J_{l_{1}}(p_{1}) &   &   & 0 \\
    & \ddots &   &   & \\
     &   & \ddots &   &  \\
  0 &   &    & J_{l_{s}}(p_{s})
  \end{matrix}
  \right]_{n\times n}
$$
%matrix end

\subsection{§9 初等因子组}

\paragraph{定义}

重集(multi-sets)-集合中相同的元素允许出现若干次

\paragraph{定义}

$\mathcal{A} \in \mathcal{L}(V)$,
$V=V_{1}\bigoplus \cdots \bigoplus V_{l} (*)$,\\
其中$V_{1},\cdots,V_{l}$是$\mathcal{A}$-不可分的,设$A_{i}=\mathcal{A}\arrowvert_{V_{i}},I=1,\cdots,l$,则重集$\{ \mu_{\mathcal{A}_{1}},\cdots,\mu_{\mathcal{A}_{l}}\}$称为$\mathcal{A}$关于$(*)$的初等因子组.

\paragraph{目的}

(1) 证明初等因子组由$\mathcal{A}$确定,与$V$的$\mathcal{A}$-不可分子空间的直和分解无关
(2) 通过初等因子组可以“唯一”地确定$Jordan$标准型

\paragraph{引理9.1}

设$\mathcal{A} \in \mathcal{L}(V),V=F[\mathcal{A}] \cdot \vec{v},\mu_{\mathcal{A},\vec{v}}=pq$,其中$p,q \in F[t] \backslash F,$首一,令$\vec{w}=q(\mathcal{A})(\vec{v})$,则$\mu_{\mathcal{A},\vec{w}}$.

\paragraph{引理9.2}
设$\mathcal{A} \in \mathcal{L}(V),V=F[\mathcal{A}]\cdot \vec{v}$,设$\mu_{\mathcal{A}} = p^{m}$,其中$p\in F[t],$则$\forall k \in \mathbb{N}$\\

$$
rank(p(\mathcal{A})^{k})=
\begin{cases}
(m-k)deg(p) &  0 \leq k < m\\
0 & k \geq m
\end{cases}$$

\paragraph{引理9.3}
%设$\mathcal{A} \in \mathcal{L}(V),f \in F[t]$,如果$U \subset V$是$\mathcal{A}$-不变的,则$U$也是$f(\mathcal{A})-$不变的.
设$\mathcal{A} \in \mathcal{L}(V),f \in F[t]$,如果$U \subset V$是$\mathcal{A}$-不变的,则$U$也是$f(\mathcal{A})-$不变的.

\paragraph{引理9.4}
如上假设,再令$V =  U_{1} \bigoplus \cdots \bigoplus U_{l}$,其中$U_{1},...,U_{l}$是$\mathcal{A}$-不变子空间,则$f(\mathcal{A})(V) = f(\mathcal{A})(U_{1}) \bigoplus \cdots \bigoplus f(\mathcal{A})(U_{l})$



\paragraph{定理9.1}

设$\mathcal{A} \in \mathcal{L}(V),\mu_{\mathcal{A}}=p^{m},$其中$p \in F[t]$不可约,对$\forall l \in \mathbb{Z}^{+},$令$n_{l}$为$p^{l}$是$\mathcal{A}$关于某个$\mathcal{A}$-不可分子空间直和分解的初等因子组的重数.再令$r_{l}=rank(p(\mathcal{A})^{l})$,其中$l \in \mathbb{N}$,则$n_{l}=\frac{1}{d}(r_{l+1}+r_{l-1}-2r_{l})$

\paragraph{定理9.2}
设$\mathcal{A} \in \mathcal{L}(V),\mu_{\mathcal{A}}$的两两不同、首一的不可约因子是$p_{1},\cdots,p_{s} \in F[t]$,对$\forall I \in \{1,\cdots,s\},l \in \mathbb{Z}^{+}$,令$N(i,l)$是$p^{l}_{i}$在$\mathcal{A}$的某个初等因子组中的重数,$R_{i,l} = rank(p_{I}(\mathcal{A})^{l})$,则$N(i,l)= \frac{1}{deg(p_{I})}(R_{i,l+1}+R_{i,l-1}-2R_{i,l})$

\subsection{§10 Jordan标准型的唯一性和应用}


\ifx\total\undefined
\end{document}
\fi