\documentclass{ctexart}
\usepackage{geometry}
\usepackage{amsmath}
\usepackage{amsfonts}

\begin{document}
\section{第二章}
\subsection{§5 特征子空间的应用}
\subsubsection{§5.1 线性算子和矩阵的对角化}
\paragraph{定义}
设$\mathcal{A}\in\mathcal{L}(V)$,$\mathcal{A}$在F中互不相同的特征根的集合称为$\mathcal{A}$在F上的谱(spectrum)
\paragraph{定义}
设$\mathcal{A}\in\mathcal{L}(V)$,如果$\mathcal{A}$在V的某组基下的矩阵是对角的,则称$\mathcal{A}$是可对角化的。设$A \in M_{n}(F)$,如果A相似于某个对角矩阵,则称A在F上是可对角化的。
\paragraph{定理5.1}
设$\mathcal{A}\in \mathcal{L}(V)$,则下列断言等价:\\
(i)$\mathcal{A}$可对角化\\
(ii)$\mathcal{A}$有n个线性无关的特征向量,其中n=dim(V)\\
(iii)V=$\bigoplus_{\lambda \in spec(\mathcal{A})}$\\
\paragraph{推论5.1}
设$\mathcal{A}\in\mathcal{L}(V)$,dimV=n,如果$\mathcal{A}$在F中有n个互不相同的特征根,则$\mathcal{A}$可对角化.
\paragraph{定理5.2}
设$\mathcal{A}\in\mathcal{L}(V)$,则$\mathcal{A}$可对角化$\Leftrightarrow$(i)$\mathcal{X}_{\mathcal{A}}$在F中可以分解为一次多项式之积(ii)$\mathcal{A}$在每个特征根的代数重数与几何重数相同.
\subsubsection{§5.2 复数方阵的三角化}
\paragraph{引理5.2}
设V是$\mathbb{C}$上的n维线性空间,n>0,$\mathcal{A}\in\mathcal{L}(V)$,则$\mathcal{A}$有n-1维不变子空间.
\paragraph{定理5.3}
设$\mathcal{A}\in\mathcal{L}(V)$,其中V是$\mathbb{C}$上n维线性空间,则存在V中一组基,使得$\mathcal{A}$在该基下的矩阵是上三角型的.
\paragraph{推论5.2}
设A$\in M_{n}(\mathbb{C})$,则A相似于一个上三角型矩阵.
\paragraph{引理5.3}
设$\mathcal{A}\in\mathcal{L}(V)$,U是$\mathcal{A}$-子空间,定义: \\
$\overline{\mathcal{A}}:V/U \to V/U$\\
$\vec{a}+U\mapsto\mathcal{A}(\vec{a})+U$\\
则$\overline{\mathcal{A}}\in\mathcal{L}(V/U)$
\paragraph{定义}
设$\mathcal{A}\in\mathcal{L}(V)$,U是$\mathcal{A}$-子空间,则\\
$\overline{\mathcal{A}}:V/U \to V/U$\\
$\vec{v}+U\mapsto\mathcal{A}(\vec{v})+U$\\
称为$\mathcal{A}$关于U的商算子.
\paragraph{命题5.1}
设$\mathcal{A}\in\mathcal{L}(V)$,U是$\mathcal{A}$-子空间\\
$\Pi : V \to V/U$ 自然投射\\
则(i)  $\Pi \circ \mathcal{A} = \overline{\mathcal{A}} \circ \Pi$,其中$\overline{\mathcal{A}}$是$\mathcal{A}$关于U的商映射.\\
(ii) 设$\varphi : V/U \to V/U$满足$\pi \circ \mathcal{A} = \varphi \circ \pi$,则$\varphi = \overline{\mathcal{A}}$
\paragraph{定理5.3}
设V是n维线性空间,n>1,设$\mathcal{A}\in \mathcal{L}(V)$,U是$\mathcal{A}$-子空间,d=dimU>0,设$\vec{e_{1},...,\vec{e_{d}}}$是U的基,$\vec{e_{1}},...,\vec{e_{d}},\vec{e_{d+1}},...,\vec{e_{n}}$是V的基.记A|$_{U}$为$A_{U}$,$\mathcal{A}$关于U的商算子为$\overline{\mathcal{A}}$.令$A_{U}$为$\mathcal{A_{U}}$在$\vec{e_{1}},...,\vec{e_{d}}$下的矩阵.$\overline{A}$为$\overline{A}$在$\vec{e_{d+1}},...\vec{e_{n}}$下的矩阵,则$\mathcal{A}$在$\vec{e_{1}},...,\vec{e_{d}},\vec{e_{d+1}},...,\vec{e_{n}}$下的矩阵是\\
%matrix begin
\begin{equation}
A=
\left[ \begin{array}{cc}
A_{U} & B\\
0 & \overline{A}
\end{array} 
\right]
\end{equation},其中B$\in F^{d\times(n-d)}$
%matrix end
\paragraph{推论5.2}
沿用定理5.3中记号,$\mathcal{X_{\mathcal{A}}}(t)=\mathcal{X_{\overline{A}}}(t)\mathcal{X_{\mathcal{A_{U}}}}$
\paragraph{命题5.2}
设$\mathcal{A}\in\mathcal{L}(V)$.U是$\mathcal{A}$-不变子空间,P$\in F[t]$则\\
(i) U是$\mathcal{P}(\mathcal{A})$-子空间\\
(ii) 设$\overline{\mathcal{A}}$和$\mathcal{\overline{P(\mathcal{A})}}$是$\mathcal{A}$和$\mathcal{P}$($\mathcal{A}$)关于U的商算子,则$\mathcal{P}$($\overline{\mathcal{A}})=P(\overline{A})$
\end{document}
